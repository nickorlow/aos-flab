\usepackage{listings}
\begin{document}

\title{CS 380L Lab 0}
\author{Nicholas Orlowsky}
\maketitle

\section{Background}

\subsection{Machine Specifications}

This test was conducted on a desktop computer with an AMD Ryzen 5 1600U (x86_64) which 
has 6 cores clocked at 3.2 GHz. The sytem has 16GB (2x8 GB) of DDR4 memory operating at 
3066 MHz.

The system is running Arch Linux with Linux Zen kernel version 6.10.6. Debugging Kernel 
was ran on qemu version 9.0.2.

\section{Results}

\subsection{Boot Time}

\subsection{Detected Devices} 

\subsection{Kernel Debugging}

CR3 never changed. Adding nopti to the command line did not change this behavior, however 
adding pti=on does enable it. This is because nopti is an alias for pti=off, and by default 
it's inferred to be pti=auto, and auto does not enable it on AMD CPUs because it's a security 
option only necessary for Intel processors.

\subsubsection{QEMU Command Line}

qemu-system-x86_64 \
    -enable-kvm \
    -m 4096 \
    -nic user,model=virtio \
    -kernel /home/nickorlow/programming/school/aos-lab0/build/arch/x86_64/boot/bzImage \
    -nographic \
    -drive file=$1,media=disk,if=virtio \
    -append "root=/dev/vda1 console=ttyS0,115200n8" \
    -snapshot \
    -monitor telnet:127.0.0.1:55555,server,nowait \
    -s

\subsubsection{Build Configuration Flags}

\underline{CONFIG_DEBUG_INFO y}: This flag generates debugging symbols for the kernel, 
allowing it to be more transparently debugged with gdb.

\underline{CONFIG_DEBUG_INFO_DWARF4 y}: Specifies that debug information generated 
should be in the DWARF4 format.

\underline{CONFIG_GDB_SCRIPTS y}: Creates GDB helper scripts to help debug kernel

\underline{CONFIG_GDB_INFO_REDUCED n}: Increases generated debugging information by 
not generating less information for structure types. 

\underline{CONFIG_KGDB y}: Configures kernel for remote GDB debugging

\underline{CONFIG_FRAME_POINTER y}: Enables frame pointers for increased debugging 
information

\underline{CONFIG_SATA_AHCI y}: Enables AHCI Serial ATA, making it so that 
your SATA disk driver doesn't need to be loaded as a module

\underline{CONFIG_KVM_GUEST y}: Enables optimizations for when the kernel is running 
as a guest to KVM.

\underline{CONFIG_RANDOMIZE_BASE n}: Disables security protections which randomizes the 
memory location of the kernel to make debugging easier. This option is usually enabled 
as a security feature.

\underline{CONFIG_SMP y}: Enables support for multiple cores

%\underline{DEBUG_INFO_COMPRESSED_NONE y}: Enables support for multiple cores
%\underline{DEBUG_INFO_SPLIT y}: Enables support for multiple cores
%\underline{KGDB_HONOUR_BLOCKLIST y}: Prohibits unsafe breakpoints
%\underline{KGDB_SERIAL_CONSOLE n}: Disables GDB over serial
%\underline{KGDB_TESTS n}:  Disables KGDB test suites
%\underline{KGDB_LOW_LEVEL_TRAP n}: Disallows debugging with traps in notifiers
%\underline{KGDB_LOW_LEVEL_TRAP n}: Disallows debugging with traps in notifiers

\section{Conclusion}

%\section Citation}
% https://cateee.net/lkddb/web-lkddb/KGDB.html

\end{document}
